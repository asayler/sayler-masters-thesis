\chapter{Introduction}
\label{chap:intro}

Data is everywhere. Our devises produce it. Our web sites consume
it. Governments collect it and business request it. But in this ever
present whirlpool of data exchange, how can we stay in control of our
data? How can we ensure that those who we wish to can access it can
while preventing those who we do not from doing the same?

Fortunately, there are methods for securing our data: strong
cryptography systems like AES or RSA are perfectly capable of allowing
us to control exactly who can read our data. Unfortunately, these
systems are often difficult if not impossible for the average end user
to employ properly. Other times, they are simply treated like ``magic
fairy dust'' to be applied to various products in the name of
``just-add-crypto'' security with little heed paid to the security of
the implementation or components of the system.

How can we make encryption more usable? How can we make it more
accessible? And how can we accomplish both while maintain capability
with an array of modern use cases involving sharing, syncing, and
growing demand? Custos aims to provide an answer to these questions by
providing a secure key-value store that can be used to implement a
key-storage as a service platform.

\section{Overview}

At it's core, Custos is just another key value store. Actually, it's
not even that. It's just a wrapper around one of several existing
key-value stores. It is not the method of key-value storage that makes
it unique. Instead, it is it's ability to provide flexible,
fine-grained access control to key-value pairs that make it
interesting. These access control capabilities make Custos an ideal
system for implementing a secure encryption key storage service. For
it is not encryption itself that leads to usability issue, but the
need to secure manage and store encryption keys. Custos aims to solve
the key-storage problem inherent in many modern applications on
cryptographic security. In this section, we'll discuss the basics of
the Custos rational and goals.

\subsection{Separating Functionality from Trust}

Data security is an issue of trust. Who do we trust to access our
data? Who do we trust not to misuse it? Who do we trust not to share
it without permission? Today, we have very little practical ability to
make decisions regarding who we should trust with our data. DO you
want to use Facebook to communicate with family and friends? Great,
but you must trust Facebook with your personal data. Want to use Gmail
for it's sleek web interface and cloud-based accessibility? Fine, but
you must trust Google with all of your email. Sure, you could forgo
Facebook or Google or any of a wide variety of web services to avoid
trusting them with you data, but as you drift toward the hermitage of
self-imposed digital exile, the last of your former friends slowly
fading from memory as they cease to even recall your existence absent
their normal methods of web based contact, fumbling through the
vestigial pages of a phone book vainly hoping to find a number for
someone's cell phone that has never, and will never, be listed there,
you may decide that giving up control over whom you trust with your
data is a perfectly fair price to pay to rejoin the 21st century land
of living, breathing, digitally exposed souls.

And even if you could live without modern cloud-based services, even
good old fashioned computing technology involves placing trust in
systems or parties beyond our control. Do you trust your computer
manufacture not to have installed a hardware key logger that sends
back data to it's parent? Do you trust your operating system not to
have a government-mandated back door for covert third party access?
Do you trust yourself not to loss your laptop, exposing all of the
data on it to whomever might find it?

This is the crux of the problem. In order to benefit from many of the
modern features and amenities of the digital world, you must pay the
entry price of deference of trust to organizations, technologies, and
individuals whether you would like to or not.

So how can we solve this problem? This disconnect between the services
we desire and the trust we'd prefer not to cede? It seems unlikely
that we can eliminate trust form the equation all together. Systems
are too fragile and technology tied to human action; we will always
require some level of trust in some part of the system we rely on.

We might not be able to remove trust, but what if we could at least
isolate it. Separate trust from features. Disentangle what we use from
who we trust. What if we could have one company we trusted with
storing and controlling access to our data and another company we
relied on to take that data provide us with a useful service using it:
the ability to use Facebook or Google without trusting (or at least
unrestrictedly trusting) Facebook or Google.

With such an ecosystem, we might be able to rely on markets or similar
means to provide us with the basic platform for securing our
data. Trustworthiness would become a service; a commodity to be bought
and sold. We could chose and pay the companies responsible for
securing our data based on their level of trustworthiness, while
choosing and paying the companies that use our data and us it provide
us with relevant features on the basis of the feature they
provide. This would remove the current coupling of features and trust
we see today, a coupling that often leads to a conflict of interest
between the features we desire and the trust we're willing to
provide. Instead we'd assign trust on the basis of perceived
trustworthiness while selecting untrusted services on the basis of
feature sets; the trusted party acting as a gatekeeper between the
untrusted party and our data. We could even distribute trust across
multiple parties to avoid having to trust any single party
completely. Such a decoupling of trust and services would provide a
lot of flexibility to maximize both the security, and the utility, of
our data.

\subsection{Key Storage as a Service}

Encryption provides the

Discuss separating key storage from encryption.

UUID : Data Encryption Key

\subsection{Flexibility and Usability}

Discuss need for variable levels of trust and flexible methods of
authentication. Discuss fitting security to desired uses. Avoiding
``one size fits all'' mentality.

\section{Background}

\subsection{Encryption}

Discuss symmetric and asymmetric encryption, uses, strengths,
weaknesses, etc.

Encryption = solved problem

Key Storage = The real challenge

\subsection{Secure Storage}

Discuss existing approaches (full stack systems, layered systems),
strengths, weaknesses, etc

\subsection{Human Factors}

Disuses existing usability challenges

\section{Related Work}

\subsection{Authentication Systems}

PAM, Shibboleth, OAuth, OpenID, SAML, Etc

\subsection{File Systems}

eCryptFS, LUKS, Oceanstore, Tahoe, AFS, Etc

\subsection{Other Systems}

Anything else?
