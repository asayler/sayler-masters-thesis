\chapter{Conclusion}
\label{chap:conclusion}

Today, more so than ever, having secure, usable ways of protecting our
data is of the utmost importance. Unfortunately, existing technologies
are failing us. Most data storage system provide little security be
default. The systems that due exist to secure data like strong
cryptography are challenging to use, especially in a manner that fits
our modern usage habits. Furthermore, we have been lead to conflate
features with trust, or at least to be willing to forgo trust in favor
of features. How can we protect and control our data in a manner that
is easy to use, well suited to modern user cases, and allows us to
control whom we chose to trust with our data while still leveraging
the rich set of features available on modern data-driven services?
Custos aims to provide the basis of one possible answer to these
questions.

Custos's primary contributions include:

\begin{packed_desc}
\item[Trust-Separation Architecture:] For creating dedicated secret
  stores run by trusted providers
\item[``Key Storage as a Service Platform:] For providing a dedicated
  encryption key storage and access control service
\item[A generic, flexible access control scheme:] Including the use of
  access control chins for generalized access control specifications
  using a range of authentication primitives
\item[Custos protocol]: For standardizing the exchange of secrets and
  the authentication information required to access them
\item[Custos Server Design and Implementing:] Demonstrating the basics
  of Custos server design
\item[Several Proof-of-Concept Custos Applications:] Demonstrating the
  range of Custos use cases and the manner in which one might
  interface with a Custos server.
\end{packed_desc}

\section{Conclusions}

Custos provides a flexible, secure key:value storage
architecture. This architecture can be used to store data directly,
regulating access via flexible authentication primitives. It can also
be used as a component in a larger encryption-based data security
scheme, where Custos provides a logically centralized ``Key Storage as
a Service'' platform capable of making the associated encryption
systems more flexible, and thus more usable. The flexibility of the
Custos architecture allow it to solve a variety of modern security
problems, and forms the basis for a Trust as a Service security model
appropriate for a range of contemporary applications.

\subsection{Successes}

My experience with thus far has underlined the flexibility of
system. Custos's flexibility is primly derived from two attributes of
the system: Custos's logically centralized nature, and Custos's
authentication and access control scheme. The centralization of Custos
allows many services to rely on it as a standardized security and
access control system. The centralized nature also supports
multi-user, multi-device use cases that are not possible in local,
ad-hoc secret storage systems. The Custos access control scheme, from
the pluggable authentication modules supporting a range of
authentication parameters to the arbitrary access control chains,
allows for a wide range of access control intentions to be expressed
in a common, easy to use, language.

Custos flexibility, in turn, enable usability improvements across a
range of application domains. Whether it's allowing for encryption
system that can function across our myriad of modern devices and allow
us to securely share data with other users, or providing us with a
centralized personal data repository to which we can provide
controlled access for the web services of our choosing, Custos's
encourages secure designs that also remain usable. While I have not
formally verified this usability, anecdotal experience using Custos to
protect my own data has shown it to enable secure systems that would
not otherwise be possible.

Finally, Custos provided a trust-separation system with a well defined
interface. This could form the basis of market-derived approach to
security. Once we can separate trust from unrelated functionality, it
becomes far easier to select and reward dedicated ``trust providers''
on the basis of their trustworthiness, while still retaining the
ability to select untrusted services on the basis of the features they
provide. If we truly wish to protect our data, we must create an
incentivizes system for proving trustworthiness as a service to users,
just as we currently have systems that incentivizes providing features
to users. Custos provides a standardized, inter-provider compatible
architecture on which such an ecosystem could be built.

\subsection{Challenges}

While Custos has had some successes, it is certainly not without its
challenges. First and foremost, Custos does not eliminate the need for
trust, it simply isolates and it and makes it more flexible. In order
to gain many of the usability benefits of a Custos-backed encryption
system, you must still trust a third party. While it is possible to
operate your own Custos server, and some may very well do that, doing
so is not within realistic reach of many end-users, and may very well
negate many o the usability benefits Custos provides. Trust is still
necessary, Custos only helps regulate it.

Furthermore, Custos, like any new system, faces adoption
challenges. At this point in time, there are no mainstream systems
that utilize Custos. While I believe many systems could be refactored
to use Custos with minimal effort, the task of doing so, or convincing
other to do so, in still non trivial. In order for Custos to succeed,
it must see a least moderately widespread standardization and
adoption. There must be readily available security systems that
utilize Custos. And Custos providers must be easily found, affordable,
and numerous (in order to ensure some modicum of trustworthiness
through competition for your perception of their
trustworthiness). Until Custos, or a similar standard secret storage
system and provider bases exists, using Custos in live production
settings will remain challenging.

The performance overhead of using Custos is also not well
understood. For many end-user applications, raw performance is not the
primary concern, and there may exist a willingness to sacrifice some
performance in the name of increased security. That said, the overhead
of Custos across a range of applications has not been evaluated. Nor
have Custos performance bottlenecks and possible improvements been
identified or proposed. This will have to come with time and
additional use of Custos and Custos-backed applications.

Custos's authentication system, especially the access control chain,
component is highly flexible. But it remains to be seen whether or not
this flexibility will lead only to increase ease of use (a good
thing), or whether it risks giving the user too much freedom making it
prone to misconfiguration and errors. Furthermore, the Custos protocol
is believed to be capable of supporting a wide range of authentication
primitives, but at this time, only a handful of authentication
primitives have actually been tested. Whether or not the current
format is capable of supporting more complex authentication schemes,
and how easily they might be implemented within the Custos plugin
framework, remains to be seen.

\section{Future Work}

The Custos's work presented in this document represents the
culmination of the initial Custos design and implementation effort. It
has resulted in a usable secret storage services and the basis of a
variety of applications that leverage this service. That said, there
is plenty of work to be done to make Custos a fully production-ready
and proven system.

One of the key tenets of Custos design was usability, both the base
usability of Custos itself, and the increased usability of
applications leveraging Custos. I would like to conduct one or more
user studies evaluating the usability of Custos and Custos backed
applications. This might include measuring the success user have
building access control chins that meet their intentions (vs those
that subvert intentions through misconfiguration). It would also
likely include measuring the usability difference between a
traditional encryption system and a Custos-backed encryption
system. Backing up the Custos design principles with some solid
usability data, and adapting these principles where necessary, is a
high priority for future Custos research.

I would also like to expand the Custos server implementation prompt,
making it more robust, scalable, and widely deployable. This will
include switching to a high performance key:value back-end, improving
the Custos authentication plugin interface, producing plugins for
additional authentication primitives, and improving the efficiency of
the Custos access control chain verification process. I would also
like to explore availability and redundancy features of the Custos
server.

I plan to build out several Custos-backed applications. This may
include wither new native applications (like a Dropbox\cite{dropbox}
encryption plugin) or the modification of existing applications (like
eCryptfs~\cite{eCryptfs}). These applications would allow further
testing of the Custos architecture and server in a production
environment, and might enable some of the use cases studies previously
mentioned.

Finally, Custos deserves a formal security audit to fully evaluate the
security of the server, client libraries, and communication
protocol. If Custos is to interact with secure systems, it must be
usable without jeopardizing the security of these systems. Fortunately
the Custos code base is still small enough that a manual audit is
possible. Subjecting Custos to automatic auditing tools or
bounty-based exploit contests might also yield interesting results
with respect to the security of the underlying systems. Issues and
exploits discovered in such an audit would be addresses in the design
and implementation of Custos.

\section{Discussion}

The cryptographer Bruce Schneier once wrote, ``It is insufficient to
protects ourselves with laws; we need to protect ourselves with
mathematics''~\cite{schneier1996applied}. He is, of course, referring
to strong cryptography as the great technological equalizer, allowing
anyone with access to a computer to achieve ``the same security as the
largest governments''~\cite{schneier2000secrets}. As it turns out,
this is not true\footnote{Schneier acknowledges the naivety of his
  original Utopian outlook on cryptography in his more recent
  works~\cite{schneier2000secrets}.}.

Not only has cryptography failed to enable the average person to
protect herself~\cite{green-challenge}, the belief that it can has led
to an increasing gap between the average user, who desires her data to
be protected but who lacks the ability to protect it, and the elites
who are capable of protecting their own data while also preying on
those who can't. In his counter-culture manifesto, \textit{Computer
  Lib}, Ted Nelson states that, ``Guardianship of the computer can no
longer be left to the priesthood''~\cite{nelson1972computer}. It is
just as true today as it was when he wrote it. We can not afford to
forgo control of our data, leaving it to be picked over by the
``priesthood'' of crypto elite. We have already seen where that has
taken us in the revelations of Mr. Edward Snowden,
et. al.~\cite{GreenwaldPrism}: to a world where governments and
criminals (and governments turned criminals) will prey on the average
computer user who lacks the tools to adequately protect herself.

Custos is not about the mathematics. It is about making the
mathematics, and the resulting encryption techniques, available to
everyone. It is about increasing security through the commoditization
of trust, through an increase in usability, and through the
flexibility to support a diversity of end user intentions. We all want
to protect our data, and Custos aims to provided the basis for a set
of tools that will allow us to do that.

Security researchers can no longer afford to ignore the big
picture. Encryption alone is useless. Security requires a holistic
treatment. We must approach security from a technological, legal,
social, and anthropological standpoint. Only then we consider all of
these factors can we hope to make systems truly secure. Custos is one
attempt to accommodate a wider outlook in order to increase end user
security. I hope that other such project will follow and that Custos
and related efforts will succeed. We must reclaim security, reclaim
encryption, and reclaim control of our data. Our continued digital
freedom, and by proxy, our physical freedom, depend on doing just
this.

%%  LocalWords:  Schneier Snowden et al
