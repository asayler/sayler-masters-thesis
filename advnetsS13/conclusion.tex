\section{Conclusions and Future Work}
\label{sec:conclusion}

Encryption key storage is too tightly coupled with existing encrypted
file system solutions. This tight coupling makes these systems
inflexible and thus unsuitable for many of our required use cases.
Such inflexibility leads to the underutilization of data encryption by
vast portions of the user base at a time when encryption is becoming a
more and more important tool for retaining control of our data. We
believe that this issue can be solved by providing a ``Key Storage as
a Service'' system that separates encryption key storage and access
control from the underlying encryption mechanisms that use these
keys. Custos provides this flexible Cloud-based key storage service.

Going forward, we hope to expand or prototype to provide example
Custos integrations with a variety of existing secure file systems,
from systems like eCryptfs to systems like OceanStore. We plan to use
these implementations to demonstrate how Custos can extend the
flexibility and usability of a variety of encryption solutions across
a variety of use cases. We also plan to further analyze the threat
vectors inherent to a ``Key Storage as a Service'' system and provide
suggestions for practically mitigating these threats in Custos
deployments. We have plans to explore a multi-provider Custos system
that reduces the trust required for each Custos server in favor of
distributing portions of each key across multiple servers. We hope to
investigate the usability of Custos in a variety of real-world use
cases.
